\documentclass[11pt,english]{article}
\usepackage[utf8]{inputenc}
\usepackage{babel}
\usepackage{xcolor}
\usepackage{url}
\usepackage{csquotes}
\usepackage{pgfplots}
\usepackage[style=numeric-comp]{biblatex}
\addbibresource{Resources/bibliography.bib}


%%% Custom definitions %%%
% Shorthands
\newcommand{\ie}{i.\@e.\@}
\newcommand{\eg}{e.\@g.\@}
\newcommand{\wrt}{w.\@r.\@t.~}
\newcommand{\idea}[1]{\color{blue}{#1}\color{black}}
\newcommand{\temp}[1]{\color{green}{#1}\color{black}}


\title{Medical Machine Learning technologies as an example for necessary ethical trade-offs in ML}
\author{
    Albrecht, Thomas (5733587)
    \and
    Petruck, Julian (3857386)
    \and
    Jaques, Arthur (5998179)}



\begin{document}

\maketitle

\begin{abstract}
	We use the application of Machine Learning to healthcare as a case study of ethical trade-offs.
	We concentrate on trade-offs between privacy and predictability in the use of patients' data, between group fairness and individual fairness in the attempt to make ML-based systems ``fair'', and between fairness and prediction accuracy when applying fairness constraints to the ML systems.
	Firstly, we examine and discuss whether those trade-offs are unavoidable, and relate them to moral dilemmas in moral philosophy.
	Secondly, we examine the results that are obtainable with regards to those trade-offs (where do we want to lie on the Pareto frontier?).
	In the case of the trade-off between group fairness and individual fairness, we dive into the conflict between the aggregate and the individual, between the population level view of the ``average man'' and the concrete individuals that are affected by the ethical policies.
	In our critical analysis, we relate the existing best practices in medicine and their existing literature (as an example, the four principles proposed by Beauchamp and Childress), and the fairness tools and analyses provided by the ML community.
	As a consequence, we suggest what the communities could learn from each other and what differences need to be resolved.
\end{abstract}


\setcounter{tocdepth}{2}
\tableofcontents



\section{Introduction}
% why medicine is an interesting field to clarify ethical problems, particularities
\paragraph{The intersection of machine learning and medicine}
\begin{itemize}
    \item ML is great, blah blah, historical setting, current application.
    \item Fairness problems identified in the literature.
    \item Trade-offs and moral dilemmas: algorithmic, philosophical, medical view.
    \item Peculiarities of health care as application field:
        inevitable moral dilemmas, impossibility of the ``do nothing'' solution, developed moral literature, high stakes, less readiness to sacrify performance, human comparisons.
\end{itemize}

% research questions
\paragraph{Research questions}
\begin{itemize}
    \item (Where) Are trade-offs necessary? Are algorithmic trade-offs and moral dilemmas different?
    \item What are the current results in ML? Are they going in the right direction?
    \item How are trade-off situations currently handled in medical practice? What are hidden questions?
    \item Are fairness problems of ML applied to medicine new problems intrinsic to the technology, or are they inherent to medical practice?
    \item Is ``doing nothing'' really an acceptable solution?
    \item Can we implement biomedical principles in ML?
    \item What can ML learn from medical ethics?
\end{itemize}

% methods/sources
\paragraph{Methods and sources}
\begin{itemize}
    \item Literature search from different sources: philosophy, medical ethics, fair-ML.
    \item Theoretical reflections and linking literature sources and fields; interdisciplinary connections.
    \item Work does not propose concrete implementation solutions.
\end{itemize}


\section{Trade-offs part}
\subsection{Trade-offs Intro/in general}
	\subsubsection{Define tradeoffs as  inherently contradictory(trivial for conflict as opposed to view that tradeoffs dont exist or should be avoided)}


	The notion of a tradeoff describes a decision between multiple (usually mutually contradictory) objectives, in the sense that a gain in one objective results in loss in one or more other objectives.
	The concept of trade-offs is very common in everyday life, \dots

	\dots
	
	But in economics especially, trade-offs play a huge role and there have been multiple approaches to formalize them.

	\subsubsection{First derive mathematical formulation used in economics(pareto efficiency and multi objective optimization)}

	To approach choosing a feasible decision (allocation) for various types of trade-off we will introduce multi-objective optimization.
	A general multi-objective optimization problem can be written in the following way:
	$$min(u_1), \quad s.t. x\in X$$
	Here $X$ denotes the set of all feasible decisions and $u_i(x)$ the utility/objective function for each dimension.
	For a non-trivial multi-objective optimization problem it is not possible to minimize every single objective function at the same time.
	Thus the notion of Pareto optimality is introduced:
	A decision $x\in X$ is said to Pareto dominate another solution $x'\in X$ if the following both hold:
	$$1.\quad u_i(x)\le u_i(x'), \forall i\in {1,2,\dots,k}$$
	$$2. \quad u_j(x) < u_j(x'), \forall j\in {1,2,\dots,k}$$

	Such a decision is also called Pareto optimal/Pareto efficient.
	The set of all Pareto optimal decisions is called the Pareto front.
	If the optimization problem is two-dimensional the Pareto front can be visualized in an intuitive way:
	The objectives are the axes in a 2D plane, moving along the Pareto front showcases how increasing one objective decreases the other one[maybe add a figure here with example caption (accuracy fairness seems good)].
	Note that Pareto optimality does not ensure a morally right decision in any sense (change this).\dots

	But the concept of Pareto optimality alone usually won't actually leave us with a single optimal or "best" answer to our decision problem.
	Rather, the approach eliminates all "strictly worse" possible decisions in the feasible set and the decision maker is faced with a new problem.
	To now choose one solution from the Pareto front multiple methods can be used.
	Depending on the problem at hand the decision maker could (or rather has to) potentially incorporate additional prior information (knowledge/preference).

	\dots


	\subsubsection{Talk about tradeoffs in medicine/ethics in general}
	As already hinted to earlier, tradeoffs play a central role in medicine.
	They can appear in different areas in the field, be it at high level healthcare policy decisions or when considering treatment options for one specific patient.
	Sometimes those tradeoffs arise when the decision involves a moral or ethical dilemma:
	If for example the administration of a treatment could harm the patient as a side effect one might still choose to treat the disease if it is the lesser evil (e.g. chemotherapy[cite]).
	In this case the the physician faces multiple tradeoffs:
	He has to consider the effectiveness of the administered treatment (which is uncertain for the given patient) the likelihood and magnitude of possible adverse side-effect (which are also uncertain for the given patient).
	He also has to consider \dots
	
	By applying the concept of Pareto optimality one could even say that any decision that doesn't involve a tradeoff of some sort would be trivial to make, because it would have a unique maximum.
	Of course it would be desirable to avoid many of the tradeoffs in the sense of maximizing all objectives simultaneously, but that maximizing solution might not be in the feasible set, i.e. a possible decision at the given time.

	\dots

	

	\subsubsection{elaborate on tradeoffs in medicine and whether the mathematical formulation can/should be employed (it shouldn't be)}

\subsection{Beyond Pareto}


\section{Trade-offs in medicine}
\idea{
       \#Goal: identify ethical issues and trade-offs pre-existing the application of ML,
              describe what principles are used in deciding for the best solution,
              examine how they are dealt with currently.
       Healthcare disparities are a well-accepted reality, and ``often encompass all 5 domains of the social determinants of health as defined by the US Department of Health and Human Services (economic stability, education access and quality, healthcare access and quality, neighborhood and built environment, and social and community context)'' \cite[p.~2]{Chen2021}.
}

\subsection{Trade-offs in unassisted medicine}
\idea{
       Limited resources, uncertainty, ...
}
``Trade-off'' is a tendentially mathematical notion.
Hence, mentioning trade-offs in the field of medicine or ethics might cause defensive reactions, because of the supposed complexity of ethical problems.
Suggesting that doctors apply trade-offs in their practice is a contestable affirmation, since the nature of their ethical deliberations is necessarily partly non-mathematical.
Hence, a more appropriate term here is ethical (or moral) dilemma, which is a problem that arises when opposing values or principles co-occur \cite[p.~351]{Dijkstra2020}.
Fundamentally, however, trade-offs and practical solutions to moral dilemmas are the same thing: a decision on how much to respect principles that can not be fully respected at the same time.

\subsection{Principles of medical bioethics}
A good starting point for ethical discussions in medicine are the well-established guiding principles in biomedical ethics proposed by Beauchamp and Childress: respect for autonomy, beneficience, non-maleficience and justice \cite[pp.~344-345]{Dijkstra2020}, \cite[p.~2]{Morley2020}.

\idea{
       Describe, related to ML.
       Focus on beneficience vs non-maleficience.
}

\subsection{Pragmatism}
\idea{
       A solution has to be found, since non-action is worse than everything.
       In the face of uncertainty, leeway is left
}


\section{Combining machine learning and medicine}
\subsection{Old and new problems}
% idea: most problems derive from medicine;
%       if only inherently, they are already being dealt with (or ignored altogether).
% the biggest problem introduced by ML is the unification of knowledge and strategy
% (decisions could become less federated)

\subsection{Clarification of problems}
% the mathematical rigor of ML forces us to think about those problems;
% this is a positive feature and not a disadvantage.

\subsection{Advantage of inaction}
% positive versus negative harms: in doubt, do nothing
% this reasoning is much harder to apply to critical problems as those emerging in medicine


\section{Conclusions}
% Filler subparagraphs.
% We should clarify the research questions in the introduction and answer them here.

\subsection{Are trade-offs an inherently technical problem?}

\subsection{When is (in)action justified?}


%\section{Arthur's ideas}
%\subsection{tradeoffs, or maybe introduction?}
We identify three axes of conflict when implementing fairness into ML systems.
Firstly, assuring privacy requires modifying the data (thus removing information), which probably leads to a deterioration of prediction accuracy.
(citation required).
\idea{Non-maleficience, trade-off \cite{Dijkstra2020}.}
Secondly, the typical implementation of fairness into ML systems is done in the form of group fairness measures, \ie, requires the separation of people into groups, usually by so-called sensitive attributes.
This leads to a conflict between individual fairness, with individuals wishing to be judged independently of their group identity, and group fairness, which tries to correct for supposed historical and data biases.
It further raises constraints of group belonging and typicality (is it advantageous to be `average' in its own group?).
\idea{
    (Discuss Binns).
    Use slideset 9, slide 22 for Binns comment.
}
\temp{
    Individual justice ideas seem to go exactly in the opposite direction of "Explainable AI", since they basically say that concepts that can not be put into words should be used to base a decision.
    In general, Explainable AI requirements contrast with "AI cannot make human-like judgements".
    The elements to take into account when deciding on what metric of fairness to use are multiple.
    On the one hand, we need to decide what moral principles we want to follow, i.~e.~what we intend by equal or just treatment.
    What do we consider distributive justice?
    What is the resource that has to be distributed?
    Do we care about the end-result, or only about promising equal expectancies?
    On the other hand, we have to provide a model about the sources of unfairness in the data and model we use.
    In ML terms, we have to state our assumptions about the data-generating process.
    For example, assuming historical bias means putting into question the validity of the training labels, and hence accuracy on them as a performance measure \cite[p.~6]{Rajkomar2018}.
}
\idea{
    \begin{itemize}
        \item Why we think Binns 2020 does not cancel the problem.
                cp. ``Given the epistemic uncertainty surrounding the association between protected identities and health outcomes, the use of fairness solutions can create empirical challenges'' \cite[e221]{Mccradden2020}.
                negative legacy, labeling prejudice, sample selection bias \cite[p.~6]{Chen2021}.
        \item Specificity of medicine: groups sometimes DO matter in the prediction. ``difference does not always entail inequality. In some instances, it is appropriate to incorporate differences between identities because there is a reasonable presumption of causation'' \cite[e221]{Mccradden2020}
                Importance of the ``causal structure between latent biological factors such as ancestry and their associated diseases across ethnic subpopulations'' \cite[p.~3]{Chen2021}.
        \item ML systems have the (demonstrated in practice) potential to discriminate, even if group information is not included, through for example leakage of ethnicity, which is then used as a shortcut to make the predictions (reproducing, or even amplifying, historical bias) \cite[p.~3]{Chen2021}.
                For this reason, so-called fairness through unawareness is insufficient in non-discrimination. \cite[p.~5]{Chen2021}.
    \end{itemize}
}
Thirdly, transforming the objective from a single objective of performance to a multiple objective of performance and fairness leads to in general worst performance.
We thus arrive at a trade-off between prediction accuracy (or whatever performance measure is used: sensitivity, specificity) and fairness.
\idea{
    \begin{itemize}
        \item Specificity of medicine: allocation of physical benefits and harms. Non-maleficience?
        \item ``difference between an idealised model and non-ideal, real-world behavior affects metrics of model performance (\eg, specificity, sensitivity) and clinical utility in practice.'' \cite[e221]{Dijkstra2020}.
    \end{itemize}
}

\idea{
    Why not concentrate on one tradeoff? All must be approached when the solution is implemented.
    Must be considered together, since they are not orthogonal axes. Eg., privacy might mean reducing the individual even more to group characteristics.
    Cite paper combining privacy, fairness, accuracy.
    Additionally, privacy and fairness might be in conflict: targeted data collection to correct data biases ``may pose ethical and privacy concerns as a result of additional surveillance'' \cite[p.~8]{Chen2021}.
}

\textbf{Trade-offs as inevitable features of decision problems}
On a broad view, trade-offs are the basic problem of human governance.
How many resources we allocate for one problem, leaving less for another one.
(Almost) every decision has positive and negative effects.
So subjectivity is always present, and we all accept (if only implicitly) the existence of trade-offs in every decision. 
Trade-offs are intuitively understood from a young age, and encompass all human decision-making, but also biology, evolutionary theory, and more precisely the human body \cite{Launer2020}.



\subsection{Medicine}
    \idea{
        \#Goal: identify ethical issues and trade-offs pre-existing the application of ML,
                describe what principles are used in deciding for the best solution,
                examine how they are dealt with currently.
    }

    \paragraph{Pragmatism}
    \idea{
        A solution has to be found, since non-action is worse than everything.
        In the face of uncertainty, leeway is left
    }

\clearpage


\printbibliography

\end{document}