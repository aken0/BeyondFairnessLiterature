\subsection{Trade-offs in traditional medicine}
    \idea{
        \begin{itemize}
            \item Limited resources
            \item Cost vs health care quality trade-off
            \item Can the individual decide it? (Private vs public insurance).
        \end{itemize}
    }
    Mentioning trade-offs in the field of medicine or ethics might cause defensive reactions because of the mathematical flavor they carry, which clashes with the supposed complexity of ethical problems.
    Suggesting that doctors apply trade-offs in their practice is a contestable affirmation, since the nature of their ethical deliberations is necessarily partly non-mathematical \idea{citation required}.
    Hence, a more accepted term here is ethical (or moral) dilemma, which is a problem that arises when opposing values or principles co-occur \cite[p.~351]{Dijkstra2020}.
    Fundamentally, however, trade-offs and practical solutions to moral dilemmas are the same thing: a decision on how much to weight principles that can not be fully respected at the same time.
    Far from purely qualitative reasoning, a step in the quantitative dimension of trade-offs is shown for example by evidence-based medicine \cite{Launer2020}, which serves to inform decisions on what risks are to be taken with the promise of some potential benefit.

    The perhaps most obvious trade-off in the practice of medicine, that every doctor understands, is the one between the potential gains and the risked losses \cite{Launer2020}.
    In fact, one can go as far as to ``conceptualize medicine itself as the art of managing trade-offs'' \cite{Launer2020}.
    From the doctor's allocation of time to specific patients, to the risk of switching to a new potentially better treatment, to the decision of how aggressively to treat terminal patients, every hard decision a medical practitioner has to take entails a trade-off.

    While we concentrated above on the decisions the single practitioner has to make, health care itself, as part of governance, is ridden with trade-offs.
    Health care systems themselves are administrated according to risk-benefit analyses, both as part of the overall governance budget and within the system (which operations to prioritize, what costs to cover, and others) \cite{Dionne2018}. \idea{Read the article, specify.}

    Furthermore, much attention is paid to preserving privacy when using medical records and clinical data for scientific studies.
    The European GDPR is for example an important personal data protection law, that because of unclarity and unresolved legal issues often stalls scientific research and progress as a result \cite{Eiss2020}.


\subsection{Fairness in traditional medicine}
    Healthcare disparities are a well-accepted reality, and ``often encompass all 5 domains of the social determinants of health as defined by the US Department of Health and Human Services (economic stability, education access and quality, healthcare access and quality, neighborhood and built environment, and social and community context)'' \cite[p.~2]{Chen2021}.


\subsection{Solutions from medical ethics}
    A good starting point for ethical discussions in medicine are the well-established guiding principles in biomedical ethics proposed by Beauchamp and Childress: respect for autonomy, beneficence, non-maleficence and justice \cite[pp.~344-345]{Dijkstra2020}, \cite[p.~2]{Morley2020}, \cite[p.~2]{Rajkomar2018}.
    The guiding ideas of biomedical ethics can be used to assess specific applications of ML to health, for example decision support in occupational health, by considering the potential benefits and risks with respect to those principles \cite{Dijkstra2020}.

    Despite the interest of considering biomedical ethics, actual practice seems to indicate that case-by-case evaluations of the moral implications of medical decisions are more useful than principled approaches.
    Toulmin reports for example how a commission of people from different backgrounds, faced with specific practical problems, were able to reach some consensus (disagreeing at most about the degree of the decisions), all while furiously disagreeing about the principles supporting their decisions \cite{Toulmin1982}.
    Physicians typically exert their clinical judgment only after collecting a precise case history, instead of following general theoretical considerations early on \cite{Toulmin1982}.
