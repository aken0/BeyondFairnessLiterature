\idea{
       \#Goal: identify ethical issues and trade-offs pre-existing the application of ML,
              describe what principles are used in deciding for the best solution,
              examine how they are dealt with currently.
       Healthcare disparities are a well-accepted reality, and ``often encompass all 5 domains of the social determinants of health as defined by the US Department of Health and Human Services (economic stability, education access and quality, healthcare access and quality, neighborhood and built environment, and social and community context)'' \cite[p.~2]{Chen2021}.
}

\subsection{Trade-offs in unassisted medicine}
\idea{
       Limited resources, uncertainty, ...
}
``Trade-off'' is a tendentially mathematical notion.
Hence, mentioning trade-offs in the field of medicine or ethics might cause defensive reactions, because of the supposed complexity of ethical problems.
Suggesting that doctors apply trade-offs in their practice is a contestable affirmation, since the nature of their ethical deliberations is necessarily partly non-mathematical.
Hence, a more appropriate term here is ethical (or moral) dilemma, which is a problem that arises when opposing values or principles co-occur \cite[p.~351]{Dijkstra2020}.
Fundamentally, however, trade-offs and practical solutions to moral dilemmas are the same thing: a decision on how much to respect principles that can not be fully respected at the same time.

\subsection{Principles of medical bioethics}
A good starting point for ethical discussions in medicine are the well-established guiding principles in biomedical ethics proposed by Beauchamp and Childress: respect for autonomy, beneficience, non-maleficience and justice \cite[pp.~344-345]{Dijkstra2020}, \cite[p.~2]{Morley2020}.

\idea{
       Describe, related to ML.
       Focus on beneficience vs non-maleficience.
}

\subsection{Pragmatism}
\idea{
       A solution has to be found, since non-action is worse than everything.
       In the face of uncertainty, leeway is left
}
