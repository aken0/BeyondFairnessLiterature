%\subsection{Trade-offs Intro/in general}
	%\subsubsection{Define tradeoffs as  inherently contradictory(trivial for conflict as opposed to view that tradeoffs dont exist or should be avoided)}


	The notion of a trade-off describes a decision between multiple (usually mutually contradictory) objectives, in the sense that a gain in one objective results in loss in one or more other objectives.
	On a broad view, trade-offs are the basic problem of human governance.
    How many resources we allocate for one problem, leaving less for another one.
	Trade-offs are intuitively understood from a young age as they are very common in everyday life, and encompass all human decision-making.
	Biology, evolutionary theory, and more precisely the human body can be understood in terms of trade-offs \cite{Launer2020}.

	But in economics in particular trade-offs are of special interest, they are a central point of study in the field.
	Accordingly, economists have proposed multiple approaches to formalize them.
	One such approach, that is so widespread and commonly used that it can be regarded as a convention, is Pareto efficiency and the Pareto front.

	\subsection{Multi-objective optimization and Pareto efficiency}

	To approach choosing an optimal feasible decision (allocation) for various types of trade-offs we will introduce multi-objective optimization.
	A general multi-objective optimization problem can be written in the following way:
	$$min(u_1), \quad s.t.\; x\in X$$
	Here $X$ denotes the set of all feasible decisions and $u_i(x)$ the utility/objective function representing each dimension.
	For a non-trivial multi-objective optimization problem it is not possible to minimize every single objective function at the same time.
	Thus the notion of Pareto optimality is introduced:
	A decision $x\in X$ is said to Pareto dominate another solution $x'\in X$ if the following both hold:
	$$1.\quad u_i(x)\le u_i(x'), \forall i\in {1,2,\dots,k}$$
	$$2. \quad u_j(x) < u_j(x'), \forall j\in {1,2,\dots,k}$$

	Such a decision is also called Pareto optimal or Pareto efficient.
	Any Pareto optimal decision cannot be further improved for one objective unilaterally without resulting in loss in one or more other objectives.
	The set of all Pareto optimal decisions is called the Pareto front.
	If the optimization problem is two-dimensional the Pareto front can be visualized in an intuitive way:
	The objectives are the axes in a 2D plane, moving along the Pareto front showcases how increasing one objective decreases the other one

	[add a figure here with caption (accuracy fairness)].

	Note that Pareto optimality doesn't ensure anything beyond the property derived above.
	In particular it doesn't provide any guarantees about a "fair" or normative allocation or decision.

	\dots

	But the concept of Pareto optimality alone usually won't actually leave us with a single optimal or "best" answer to our decision problem.
	Rather, the approach eliminates all "strictly worse" possible decisions in the feasible set and the decision maker is faced with a new problem.
	To now choose one solution from the Pareto front multiple methods can be used.
	Depending on the problem at hand the decision maker could (or rather has to) potentially incorporate additional prior information (knowledge/preference).

	How to choose along pareto front (how to solve)

	\begin{itemize}
	\item a priori, incorporating priors like knowledge/preference/ 
	best practices/experience
	\item a posteriori
	\item other MCDM approaches
	\end{itemize}


	\subsection{Tradeoffs in ML}
	Next we are going to characterize three main fairness related tradeoffs in machine learning systems:
	\paragraph{Accuracy vs. Fairness (Cost of Fairness in binary classification)}
	The accuracy and fairness tradeoff is a 
	\paragraph{Group vs Individual Fairness}
	
	\paragraph{Accuracy vs. Privacy}

	\subsection{Talk about tradeoffs in medicine/ethics in general}
	As already hinted to earlier, tradeoffs play a central role in medicine.
	They can appear in different areas in the field, be it at high level healthcare policy decisions or when considering treatment options for one specific patient.
	Sometimes those tradeoffs arise when the decision involves a moral or ethical dilemma:
	If for example the administration of a treatment could harm the patient as a side effect one might still choose to treat the disease if it is the lesser evil (e.g. chemotherapy[cite]).
	In this case the the physician faces multiple tradeoffs:
	He has to consider the effectiveness of the administered treatment (which is uncertain for the given patient) the likelihood and magnitude of possible adverse side-effect (which are also uncertain for the given patient).
	He also has to consider \dots
	
	By applying the concept of Pareto optimality one could even say that any decision that doesn't involve a tradeoff of some sort would be trivial to make, because it would have a unique maximum.
	Of course it would be desirable to avoid many of the tradeoffs in the sense of maximizing all objectives simultaneously, but that maximizing solution might not be in the feasible set, i.e. a possible decision at the given time.

	\dots

	elaborate on tradeoffs in medicine and whether the mathematical formulation can/should be employed (it maybe shouldn't be)
