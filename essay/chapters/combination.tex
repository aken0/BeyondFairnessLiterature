\subsection{Fairness problems at the intersection of Machine Learning and medicine}
    Ethical questions about algorithmic unfairness are a type of normative concerns \cite{Morley2020}.
    The general concern is that CDSS trained on unbalanced or biased datasets might pick up the wrong patterns and exacerbate existing inequalities in health by overfitting on advantaged groups \cite{Morley2020}\cite{Chen2021}.
    The problem is often identified in the data used for training, which might contain label prejudice (a kind of negative legacy), variability amongst clinicians and institutions, and evolving clinical knowledge \cite{Chen2021}.
    The data at deployment time is also identified as a source of problems: population shift makes the developed model not adapted to the current population, and the usual lack of ground truth labels at test time makes evaluation difficult \cite{Chen2021}.
    Ethical discussion emerges in classification problems, when the operating point of the algorithm is chosen \cite{Fletcher2021}.
    This corresponds to the trade-off between false positives and false negatives, a well-known question in medicine (for example in breast cancer screening \cite{Fletcher2021}).
    While bias is an ethically neutral term indicating statistical imbalances, unfairness is the judgement of bias as incompatible with moral principles \cite[p.~6]{Fletcher2021}.
    Examples of biases of interest are sampling bias, unforeseen correlations, true systemic bias with biological causes, and batch effects \cite{Fletcher2021}.
    Fairness in Machine Learning (ML) is usually defined in terms of groups, quantified by group fairness measures.
    The consensus is that fairness through unawareness is not the right solution, because of sensible attribute leakage and the true effects of latent biological factors in many diseases \cite{Chen2021}.



\subsection{Old and new problems}
    The practical problem of applying ML to health care decision tasks carries with it a certain number of unavoidable decisions about the relative importance of contrasting principles.
    In short, trade-offs have to be made.
    In the previous sections, we concentrated on trade-offs related with fairness considerations.
    An interesting aspect of such trade-offs is their origin.
    We argue that many fairness-related trade-offs originate from the decision (medical) problem itself, and not from the technology used to solve it.
    This means that the analysis of fairness is necessarily domain-dependent \cite[p.~5]{Fletcher2021}, and in our case must draw on medical ethics literature \cite[p.~2]{Morley2020}.
    In particular, ``considerations internal to medical science'' and ``contextual values'' must help inform the decision on which methodological criteria to focus \cite[p.~252]{Ho2011}, and whether the available data and ML technology allow such a focus.
    Hence, we suggest abstracting from ML itself and regard it as a general technology used to solve a pre-existing problem.
    We thus take inspiration from philosophy of technology, recognizing the interaction between our behavior and the technology we use \cite{Dijkstra2020}.
    Our reason to, on a first analysis, disregard the specifics of ML is that in many cases, ML often simply specifies existing trade-offs and makes them unavoidable.

    As a concrete example, take the problem of unbalanced training data causing the ML algorithm to reach better performance on over-represented groups.
    This problem is hidden, but still present, in unassisted medicine.
    Medical practitioners arguably learn the best treatments partly from experience.
    If the majority of their patients come from a particular group, it is very likely that they will ``overfit'' their knowledge to that group, or at least be able to predict their response to particular treatments better.
    Similarly, textbook knowledge is partly derived from observations from medical practitioners and/or statistical studies.
    Biased data informing those studies will bias the observed results \cite{Chen2021}.
    This effect is exemplified by hearth health research, where research on biased data (higher proportion of men) lead to uneven improvements in heart attacks treatment with respect to gender \cite[e221]{Mccradden2020}.
    The data that informs that knowledge is probably very similar to the data used to train ML algorithms.
    The resulting inferences will hence be similarly biased as a consequence of data imbalances.
    In this sense, ML systems even have a better potential to solve the problem, using for example importance weighting or under-/over-sampling \cite[pp.~6-8]{Chen2021}.
    By no means do we intend to suggest that the solution is easy, since blindly applied technical fixes may introduce undetected harms (contrasting with the bioethical principle of non-maleficence).
    However, ML has the potential to fix biases in a way that traditional medical practice can not \cite[p.~e222]{Mccradden2020} (although traditional statistics can help, see e.~g.~importance sampling).

    The issue of reducing an individual to a group identity already exists in statistics, and arises in classical clinical practice as well.
    For example, there is evidence for the strong moral resistance to the use of statistics (such as generalization-allowing base rates) in sensitive situations \cite{Tetlock2003}.
    Despite claims about the possibility of `personalized medicine' thanks to ML methods, individuals are still reduced to their features \cite{Dijkstra2020}.
    Although it is true that personalization might be an illusion, ML allows for more precise groupings \idea{[citation needed]}.

    Similarly, the problem of data privacy pre-exists the entry of ML in the medical field \cite[p.~346]{Dijkstra2020}.
    Privilege bias (models being developed for diseases that disproportionately affect a certain group) \cite[p.~5]{Rajkomar2018} is a problem that exists in classical statistical studies as well \cite{Jackson2019}.
    In short, the trade-offs we analyzed (group fairness and individual fairness, privacy and predictability, fairness and predictability) exist independently of the system used to make decisions.
    They are not inherent to the technology used to solve them, ML, but to the goals and requirements of the system.
    ML can be used to proactively advance health equity (beneficence), and not only avoiding harms (non-maleficence) \cite[p.~2]{Rajkomar2018}\cite{Mccradden2020}.



\subsection{Potential benefits and the sin of perfection}
    How can ML actively help advance health equity and fairness?
    Firstly, it imposes the need for precise definitions of what is meant by terms like ``discrimination'' and ``equity''.
    Secondly, it forces the developers of the system to choose precise weights for the principles that they want to respect, and explicitly accept the existence of trade-offs that are inherent to the problem.
    Thirdly, it makes the goals and evaluation metrics (and their implied definition of what a ``good'' solution looks like) clear.
    Knowing that those goals influence the results \cite{Dijkstra2020} (for example, pure efficiency potentially leads to the propagation of health inequities \cite[p.~2]{Rajkomar2018}), the importance of each objective has to be decided upon (and with it, the position on the Pareto frontier).
    To summarize these advantages, we can say that ML, despite the typical complaints about its inscrutability, in a way helps enforce the transparency of the decisions taken, by requiring the ethical position to be written down explicitly \cite{Williamson2021}.
    This is an important factor especially when comparing their use to current practice and human-centered decisions, where the practitioner's values are necessarily at least indirectly influencing their decisions, probably without being stated precisely \idea{[Citation needed]}.
    Let us reiterate that, in many cases, ML technologies would simply assist human decision-makers in existing tasks \cite[p.~2]{Morley2020}.
    For this reason, the analysis of their impact must be relative to the current human decision-making (taking a utilitarian point of view, as is typical for stochastic problems; see \cite{Hardin1989}), and not an absolute decision about whether they act `perfectly fairly' or not.
    A partial reason for public distrust of algorithmic solutions might be a wrong image of doctors as invulnerable and perfect figures, partly protected by physicians in trying to keep a ``symbolic facade of professional competence'' (while privately recognizing the risks and errors of their practice) \cite{Waring2005}.
    A comparison to current unaided (by ML) practice is however rarely done, and is sometimes even impossible to do because of the impossibility and/or illegality of collecting statistics about human deciders \cite{Williamson2021}.

    The visceral resistance to the use of any technological system that shows any behavior deemed as unjust might be stopping improvements in overall care, and can be considered problematic \cite{[Citation?]}.
    Do we want, for example, to refute to apply any system that does not lead to equalized outcomes but only equalized benefit \cite[p.~5]{Rajkomar2018}?
    How do we justify keeping the unfair status quo by avoiding solutions that would improve care in general and stratified across sensitive groups, just because those solutions do not perfectly solve the problem?
    Despite some resistance of doctors to evaluation (at least partially driven by fears of blame) \cite{Waring2005}, quantification of errors and fairness are needed to compare different solutions.
    Furthermore, the time efficiency and cost of the ML applications with respect to unaided clinicians must be taken into account.
    As discussed before, trade-offs are made about budget allocation and doctors' time prioritization, so reducing them might allow improvements in other sectors of medical by displacing the saved resources (which is a phrasing that might be more positively received than simply mentioning saved costs \cite{Tetlock2003}).
    Especially considering the benefit potential of AI in medicine, exceptionalism for the application in this field is unjustified \cite[p.~9]{Fletcher2021}.

    Since our view of the world is partial and hence stochastic, we have to accept that any decision is subject to uncertainty and hence probably imperfect.
    In particular, the inevitability of errors in the medical field (be them active or latent) is widely recognized by practitioners themselves \cite{Waring2005}.
    The advantage of actively, empirically trying a solution, despite the uncertainty about its results, is the positive feedback loop that it creates.
    If we observe the development of ML systems under this lens, we can accept that solutions will evolve over time based on the results they get and can be monitored at deployment time (failure auditing \cite{Chen2021}).
    This would be particularly welcome in medicine, which suffers from a lacking culture of error culture as compared to industries such as aviation and nuclear energy \cite{Waring2005}.
    While the common suspect for this lack is the ``culture of blame'' found in medicine, other factors can be found, such as the fear of external blame, the attempt to maintain a figure of competence, the normalization of errors, a revulsion to management, an individualistic culture, a skepticism about external non-expert observers, and collegiality \cite{Waring2005}.
    That is, the fact that we change the way decisions are made will change the underlying data distribution and offer us more insights about the real sources of group differences (instigating some kind of population-level behavioral change health \cite[p.~5]{Morley2020}).
    For example, pushing to correct for historical bias by applying equal allocation principles \cite[p.~6]{Rajkomar2018} will give us more diverse data based on which to infer the causes of past differences, and potentially reduce performance gaps \cite{Chen2021}.
    A possible solution is hence to develop the system that we deem the more appropriate with the current knowledge, accept the imperfection and improve them over time as they get results.
    ML systems are not tools that once applied will remain forever the same: they should be closely monitored and improved over time \cite[p.~7]{Rajkomar2018}, potentially reducing the impact of dataset shift with the integration of new data.
    However, the dynamics of the entire ecosystem make it very difficult to predict its evolution.
    Furthermore, since the problems that are dealt with are stochastic in nature, the developed solutions will necessarily be stochastic and entail some kind of trade-off, typical of technological innovations \cite{Hardin1989}.
    Dealing with stochastic problems requires a weighting of benefits and risks by their probabilities \cite{Hardin1989}, and statistical reasoning is exactly what ML is good at (and humans are not) \cite{Williamson2021}.
    Unfortunately, very little work has been done in ML to assess the evolution of the data distribution when decisions are taken by ML systems adjusted for fairness.
    Economics literature in affirmative action may be helpful in analyzing the problem \cite{Chouldechova2020}.
    \idea{
        Draw on ``How humans judge machines.''
    }



