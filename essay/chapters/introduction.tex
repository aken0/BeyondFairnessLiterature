% why medicine is an interesting field to clarify ethical problems, particularities
\paragraph{The intersection of machine learning and medicine}
\idea{
    \begin{itemize}
        \item ML is great, blah blah, historical setting, current application.
        \item Fairness problems identified in the literature.
        \item Trade-offs and moral dilemmas: algorithmic, philosophical, medical view.
        \item Peculiarities of health care as application field:
            inevitable moral dilemmas, impossibility of the ``do nothing'' solution, developed moral literature, high stakes, less readiness to sacrify performance, human comparisons.
            true predictors: ``difference does not always entail inequality. In some instances, it is appropriate to incorporate differences between identities because there is a reasonable presumption of causation \cite[e221]{Mccradden2020}''
    \end{itemize}
}

% research questions
\paragraph{Research questions}
\idea{
    \begin{itemize}
        \item (Where) Are trade-offs necessary? Are algorithmic trade-offs and moral dilemmas different?
        \item What are the current results in ML? Are they going in the right direction?
        \item How are trade-off situations currently handled in medical practice? What are hidden questions?
        \item Are fairness problems of ML applied to medicine new problems intrinsic to the technology, or are they inherent to medical practice?
        \item Is ``doing nothing'' really an acceptable solution?
        \item Can we implement biomedical principles in ML?
        \item What can ML learn from medical ethics?
    \end{itemize}
}

% methods/sources
\paragraph{Methods and sources}
\idea{
    \begin{itemize}
        \item Literature search from different sources: philosophy, medical ethics, fair-ML.
        \item Theoretical reflections and linking literature sources and fields; interdisciplinary connections.
        \item Work does not propose concrete implementation solutions.
    \end{itemize}
}
